\documentclass[]{AGUJournal}

\usepackage{amsmath,amssymb}
\usepackage[mathcal]{euscript}

\newcommand{\beq}{\begin{equation}}
\newcommand{\eeq}{\end{equation}}
\newcommand{\earth}{{\mathcal{E}}}
\newcommand{\sun}{{\mathcal{S}}}
\newcommand{\mqsun}{{\mathcal{M}}}
\newcommand{\mearth}{{\mathcal{Q}}}
\renewcommand{\center}{{\mathcal{O}}}
\newcommand{\peri}{{\mathcal{P}}}
\newcommand{\aphe}{{\mathcal{A}}}
\newcommand{\fpoa}{\Upsilon}
\newcommand{\Tsid}{T_*}
\newcommand{\Tmsol}{T_\mqsun}
\newcommand{\al}{\alpha}
\newcommand{\la}{\lambda}
\newcommand{\eps}{\varepsilon}
\newcommand{\cob}{{\cos\eps}}
\newcommand{\sob}{{\sin\eps}}
\DeclareMathOperator{\atan2}{atan2}
\DeclareMathOperator{\artanh}{artanh}
\DeclareMathOperator{\sgn}{sgn}
\newcommand{\dee}{\text{d}}
\newcommand{\du}{\dee u}
\newcommand{\dE}{\dee E}
\newcommand{\dM}{\dee M}
\newcommand{\dla}{\dee \la}
\newcommand{\dnu}{\dee \nu}
\newcommand{\MA}{M\!A}
\newcommand{\TA}{T\!A}
\renewcommand{\circle}{{\mathcal{C}}}
\newcommand{\commentout}[1]{}

\begin{document}

\title{Equation of Time}

\authors{Peter M. Norris\affil{1,2}}

\affiliation{1}{University Space Research Association, 
Columbia, Maryland, USA}
\affiliation{2}{Global Modeling and Assimilation Office,
NASA GSFC, Greenbelt, Maryland, USA}

\correspondingauthor{P. M. Norris,
Global Modeling and Assimilation Office,
NASA/GSFC, Code 610.1, Greenbelt, MD 20771,
USA}{peter.m.norris@nasa.gov}

\section{Introduction}

The earth rotates on its axis with a period $\Tsid$ called the {\em sidereal day} (after the Latin for ``star'', since it is the rotation period 
of the earth with respect to distant stars). $\Tsid$ is slightly shorter than the so-called {\em mean solar day}, or clock day, of duration 
$\Tmsol =$ 86400 seconds. This is because the earth is a prograde planet, that is, it rotates about its axis in the same sense 
(counterclockwise looking down on the North Pole) as it orbits the sun. Specifically, say the sun crosses the meridian of some location 
at a particular time. And imagine there is a distant star directly behind the sun at that moment. After one sidereal day the location will 
rotate 360$^\circ$ about the earth's axis and the distant star will again cross its meridian. But during that time the earth will have 
moved a small counterclockwise  distance around its orbit and so it will take a small additional rotation of the earth for the sun to also 
cross the meridian and thereby complete a {\em solar day}.

Put another way, a solar day is slightly longer than a sidereal day because the sun appears to move slowly eastward across the 
celestial sphere with respect to distant stars as the year passes. The path of this motion is called the {\em ecliptic}. Clearly, what 
governs the length of a solar day is the apparent velocity of the sun along the ecliptic, or, more particularly, the equatorial component 
of that velocity. But both the magnitude and equatorial component of the solar ecliptic velocity change during the year, the former 
because the earth's orbit is elliptical, not circular, and the latter because the earth's axis of rotation is tilted with respect to the orbital 
(ecliptic) plane. Thus the length of a solar day changes during the year. While these factors cause only a small perturbation to the 
length of the solar day (less than 30 seconds), the perturbations accumulate so that, at different times of the year, apparent solar time 
(``sundial time'') and mean solar time (``clock time'') can differ by as much as about 15 minutes. This difference is called the Equation 
of Time.

To be more rigorous, in the sequel, let $\earth$ denote the earth, $\sun$ the true sun, and $\mqsun$ a fictitious ``mean sun'' that moves 
{\em at constant eastward speed around the celestial equator}, completing a full orbit in a year, namely in the period $Y \,\Tmsol$, where $Y$ 
is the number of mean solar days in a year (e.g., 365.25). Thus, in one mean solar day, $\Tmsol$, the mean sun has moved an angle $2\pi/Y$ 
eastward. Hence, beyond one full earth revolution, period $\Tsid$, an additional earth rotation of $(\Tmsol-\Tsid) \, 2\pi/\Tsid = 2\pi/Y$ is required 
to ``catch up with the moving sun'', as described earlier. Hence $\Tmsol - \Tsid = \Tsid / Y$ and so
\beq
\Tmsol = \Tsid \, \frac{Y+1}{Y},
\label{eq:TMTS}
\eeq
a constant (near unity) multiple of the fixed sidereal day. $\Tmsol$ is the length of the solar day for the ``mean sun'', or the ``mean solar day''. 
Because it is invariant during the year, it is convenient for timekeeping, and forms the basis for ``mean solar time'', which at Greenwich is 
essentially UTC. By {\em definition}, $\Tmsol$ = 24h = 86400s. That is, what we know as ``hours'', ``minutes'' and ``seconds'',  are just 
convenient integer fractions of the mean solar day. In these units, the sidereal day $\Tsid$ is approximately 23h 56m 4s.

The solar zenith angle calculation (in {\tt MAPL\_sunGetInsolation}) needs the {\em local solar hour angle}, $h_\sun$, which is the angle, 
measured westward along the equator, from the local meridian to the sun. This is just the {\em Greenwich solar hour angle}, $H_\sun$, 
plus the longitude, so we will henceforth work exclusively with Greenwich hour angles. We should use the hour angle of the true sun, 
$H_\sun$, but a common approximation replaces this with the hour angle of the mean sun
\beq
H_\mqsun = 2\pi\,(u - 1/2),
\label{eq:HM}
\eeq
where $u$ is UTC time (in days) and the offset is needed because the mean solar hour angle is zero at ``noon''. If more accuracy is 
required, the hour angle of the true sun is typically obtained as a small correction to $H_\mqsun$ called the Equation of time, $EOT$:
\beq
H_\sun = H_\mqsun + EOT,  \quad\text{where}\quad  EOT = H_\sun - H_\mqsun.
\label{eq:EOT-def}
\eeq
As discussed above, EOT corrects for two factors:
\begin{itemize}
\item[(a)]{} the variable speed of the earth in its elliptical orbit about the sun (e.g., moving fastest at perihelion), and
\item[(b)]{} the tilt of the earth's axis of rotation with respect to its orbital plane (the ``obliquity''), which causes the equatorial projection 
of the sun's apparent ecliptic motion to vary with the season (e.g., being parallel to the equator at the solstices.)
\end{itemize}

\section{Derivation of Equation of Time}

We can write
\beq
H_\sun  = H_\fpoa - (H_\fpoa - H_\sun) = H_\fpoa - \al_\sun,
\eeq
where $H_\fpoa$ is the Greenwich hour angle of the First Point of Aries (the location of the vernal equinox, denoted $\fpoa$) and is also 
known as the Greenwich sidereal hour angle, and where $\al_\sun$ is the right ascension of the true sun (since the right ascension of 
any object is just the difference between the hour angles of $\fpoa$ and the object). Hence,
\beq
EOT = H_\fpoa - H_\mqsun - \al_\sun.
\label{eq:EOT}
\eeq
All three terms on the right of (\ref{eq:EOT}) are time variable: $\al_\sun$ changes slowly throughout the year, and is known from the earth-sun 
two-body elliptical orbit solution, while $H_\fpoa$ and $H_\mqsun$ vary rapidly with Earth's rotation. $H_\mqsun$ has a period of one mean solar 
day, $\Tmsol$, and $H_\fpoa$ has a period of one sidereal day, $T_S$.

It may seem from from (\ref{eq:HM}) that the mean sun and its hour angle are fully specified. That, in fact, is not yet the case: (\ref{eq:HM}) is really 
just a definition of UTC, namely, that one UTC day is one mean solar day and that the time of culmination of the mean sun, what we call ``noon'', 
occurs at UTC 12h00m. What we are still at liberty to do is specify the phasing of the mean sun in its equatorial orbit, e.g., by specifying the time 
$u_R$ at which the mean sun passes through $\fpoa$ (both on the equator). At this time, $H_\fpoa(u_R) = H_\mqsun(u_R)$, and so
\beq
\begin{split}
H_\fpoa(u) - H_\mqsun(u) 
& = 2\pi\, (u - u_R)\,(Y+1)/Y - 2\pi\, (u - u_R) \\
& = 2\pi\, (u - u_R) / Y\\
& = \MA(u) - \MA(u_R),
\end{split}
\label{eq:EOT-h}
\eeq
where 
\beq
\MA(u) \equiv 2\pi\, (u - u_\peri) / Y
\label{eq:MA}
\eeq
is the so-called ``mean anomaly'', known from the earth-sun two-body orbital solution, and $u_\peri$ is the time of perihelion. 
Thus, to fully determine $EOT$, through (\ref{eq:EOT}) and (\ref{eq:EOT-h}), we need only to specify $\MA(u_R)$.

To understand the mean anomaly $\MA$, consider the standard two-body earth-sun problem in which the earth $\earth$ moves in an elliptical 
orbit about the sun $\sun$ at one focus, all in the {\em ecliptic plane}. The point on the ellipse closest to $\sun$ is called the perihelion $\peri$. 
Obviously, the center of the ellipse $\center$, the focus $\sun$ and the perihelion $\peri$ are co-linear, the so-called major axis of the ellipse. 
Additionally, let $\circle$ be the circumscribing circle of the ellipse, with center $\center$ and passing through $\peri$ (and the corresponding 
aphelion $\aphe$). By Kepler's Second Law, the sun-earth vector sweeps out equal areas in equal times, so the {\em fractional area\/} of the 
elliptical sector $\peri\sun\earth$ is a linear function of time, being zero at perihelion and one a year later. Specifically, this fractional area is 
none other than the scaled mean anomaly $\MA(u) / (2\pi) = (u - u_\peri) / Y$ from (\ref{eq:MA}). Clearly $\MA(u)$ can also be interpreted as 
an angle, the angle $\angle\peri\center\mearth$ of a point $\mearth$ orbiting on the circumcircle $\circle$ at constant speed in the same direction 
as the earth, also with a yearly period, and passing through $\peri$ at the same time $u_\peri$ as the earth. Thus the point $\mearth$ can be 
conceptualized as a sort of ``mean earth'' orbiting a ``second mean sun'' (different from $\mqsun$ above) at $\center$. Note that while the 
angle $\MA(u) = \angle\peri\center\mearth$ of this mean earth at time $u$ is a linear function of time, the corresponding angle of the real earth, 
namely $\TA(u) \equiv \angle\peri\sun\earth$, called the {\em true anomaly}, is a non-linear function of time, since the real earth has a variable 
speed in its elliptical orbit, e.g., moving faster at perihelion, so that its {\em areal fraction\/} is linear in time. The relationship between $\MA(u)$ 
and $\TA(u)$ is known from the orbital solution and will be discussed later. Finally, the {\em ecliptic longitude of the earth}, 
$\la \equiv \angle\fpoa\sun\earth$ is the angle at the sun, measured in the same direction as the earth's motion, from the 
First Point of Aries $\fpoa$ to the earth. Then
\beq
\TA(u) \equiv \angle\peri\sun\earth(u) =  \angle\peri\sun\fpoa +  \angle\fpoa\sun\earth(u) = \la(u) - \la_\peri,
\eeq
where $\la_\peri = \la(u_\peri) \equiv \angle\fpoa\sun\peri = -\angle\peri\sun\fpoa$ is known as the {\em longitude of perihelion}, and is currently about
$283^\circ$, or equivalently $-77^\circ$.

With this background, we can understand the quantity $\MA(u_R)$ we are trying to specify. If we \textbf{\textit{choose}}
\beq
\MA(u_R) = - \la_\peri = \angle\peri\sun\fpoa \iff \angle\peri\center\mearth(u_R) = \angle\peri\sun\fpoa,
\label{eq:uR-choice}
\eeq
then at $u_R$, viewed from the mean earth $\mearth$, the second (ecliptic) mean sun $\center$ is in direction of $\fpoa$.
And at that same time, by definition of $u_R$, the first (equatorial) mean sun $\mqsun$, as seen from the real earth $\earth$,
is also in direction of $\fpoa$.

\section{Integrals}

Does this particular choice of $u_R$ gives zero mean $EOT$, as required for a {\em mean\/} solar time? 
Let $\langle\cdot\rangle$ denote a time average over one orbit (one year), so that (\ref{eq:EOT}), (\ref{eq:EOT-h}) and (\ref{eq:uR-choice}) yield
\beq
\langle EOT \rangle 
= \langle  \MA(u) \rangle + \la_\peri - \langle \al_\sun \rangle 
= \MA(\langle u \rangle) + \la_\peri - \langle \al_\sun \rangle,
\eeq
since $\MA$ is a linear function of $u$.
In particular, let
\beq
{\langle f \rangle}_X \equiv Y^{-1} \int_{X-Y/2}^{X+Y/2} f(u) \,\du,
\label{eq:tavg-def}
\eeq
whence
\beq
\langle EOT \rangle_{u_X} = \MA(u_X) + \la_\peri - \langle \al_\sun \rangle_{u_X}.
\label{eq:mEOT-gen}
\eeq
For example,
\beq
\langle EOT \rangle_{u_\peri} = \la_\peri - \langle \al_\sun \rangle_{u_\peri}, \quad\text{and}\quad
\langle EOT \rangle_{u_\fpoa} = \MA(u_\fpoa) + \la_\peri - \langle \al_\sun \rangle_{u_\fpoa}.
\label{eq:mEOT-eg}
\eeq
The right ascension of the true sun, $\al_\sun \in (-\pi,+\pi]$, is given by
\beq
\al_\sun = \atan2\,(\sin\la\,\cob, \cos\la),
\eeq
where $\eps$ be the earth's obliquity ($\approx 23.5^\circ$). Both $\la$ and $\al_\sun$ are zero at $\fpoa$. 
To proceed, we will use the following rate of change of $\la$ from the two-body theory:
\beq
\frac{\dla}{\du} = \frac{\dnu}{\du}
  = \frac{2\pi}{Y}\,(1-e^2)^{-3/2}\,(1 + e\cos\nu)^2,
\eeq
where $e$ is the eccentricity and $\nu \equiv \la(u) - \la_\peri$ is shorthand for the true anomaly of the earth, $\TA(u)$.
Then, without being precise on limits for now,
\beq
\langle \al_\sun \rangle
= Y^{-1} \int \frac{\al_\sun \dla}{\dla / \du} 
= \int \frac{\atan2\,(\sin\la\,\cob, \cos\la)}{(1-e^2)^{-3/2}\,[1+e\cos(\la-\la_\peri)]^2} \frac{\dla}{2\pi}.
\eeq
Finally, just as $\fpoa$ denotes the location of the vernal equinox, $\la = \al_\sun = 0$, we will also use $\fpoa'$ to denote 
the location of the autumnal equinox, $\la =  \al_\sun = \pm \pi$. In general, $u_\fpoa - u_{\fpoa'}$ is not exactly half a year.

\subsection{Zero obliquity}

For the simple case where the obliquity is zero, $\cob=1$ and $\al_\sun = \la = \nu + \la_\peri$, and so
\beq
\begin{split}
\langle \al_\sun \rangle_{u_\peri}
= & \int_{-\pi}^{+\pi} \frac{\nu+\la_\peri}{(1-e^2)^{-3/2}\,[1+e\cos\nu]^2} \frac{\dnu}{2\pi} \\
= & \frac{\la_\peri}{2\pi} \int_{-\pi}^{+\pi} \frac{\dnu}{(1-e^2)^{-3/2}\,[1+e\cos\nu]^2},
\end{split}
\eeq
since perihelion and aphelion are half a year apart by symmetry and since the $\nu$ term is odd.
The true anomaly can be expressed in terms of the {\em eccentric anomaly\/} $E \in (-\pi,+\pi]$:
\beq
\cos\nu = \frac{\cos E - e}{1 - e \cos E}
\quad\text{and}\quad
\sin\nu = \frac{\sqrt{1 - e^2} \sin E}{1 - e \cos E},
\eeq
whence
\beq
1 + e\cos\nu = \frac{1 - e^2}{1 - e \cos E}.
\eeq
and
\beq
-\sin\nu \frac{\dnu}{\dE} = -\sin E \frac{1 - e^2}{(1 - e \cos E)^2}
\implies \frac{\dnu}{\dE} = \frac{\sqrt{1 - e^2}}{1 - e \cos E}
\eeq
Hence,
\beq
\langle \al_\sun \rangle_{u_\peri}
= \frac{\la_\peri}{2\pi} \int_{-\pi}^{+\pi} (1 - e \cos E) \,\dE  = \frac{\la_\peri}{2\pi} \int_{-\pi}^{+\pi} \,\dM
= \la_\peri.
\eeq
where $M \equiv E - e \sin E$. Hence, as required, $\langle EOT \rangle_{u_\peri} = 0$ by (\ref{eq:mEOT-eg}).
Note that $M(u)$ is none other than $\MA(u)$, as per Kepler's Equation of the two-body solution.

\subsection{Zero eccentricity}

For zero eccentricity, $e=0$, we get the simple form
\beq
\langle \al_\sun \rangle = \int \atan2\,(\sin\la\,\cob, \cos\la) \frac{\dla}{2\pi},
\eeq
and, in particular,
\beq
\langle\al_\sun\rangle_{u_{\fpoa'}+Y/2} 
= \int_{u_{\fpoa'}}^{u_{\fpoa'}+Y} \hspace{-1em} \al_\sun(u) \frac{\du}{Y} 
= \int_{-\pi}^{+\pi} \!\! \atan2\,(\sin\la\,\cob, \cos\la) \frac{\dla}{2\pi} 
= 0,
\eeq
since since $\atan2$ is odd in $\la$. Then, by (\ref{eq:mEOT-gen}),
\beq
\langle EOT \rangle_{u_{\fpoa'}+Y/2} = \MA(u_{\fpoa'}+Y/2) + \la_\peri - \langle \al_\sun \rangle_{u_{\fpoa'}+Y/2} =  \MA(u_\fpoa) + \la_\peri,
\eeq
since $\fpoa'$ and $\fpoa$ {\em are\/} a half year apart for a circular ($e=0$) orbit. But also for a circular orbit, $\MA(u) = \TA(u) = \la(u) - \la_\peri$, 
so 
\beq
\langle EOT \rangle_{u_{\fpoa'}+Y/2} =  (\la(u_\fpoa) - \la_\peri) + \la_\peri = \la(u_\fpoa) \equiv 0,
\eeq
as required.

\subsection{General case}

For the general general case, 
\beq
\begin{split}
\langle\al_\sun\rangle_{u_{\fpoa'}+Y/2} 
& = \int_{u_{\fpoa'}}^{u_{\fpoa'}+Y} \hspace{-1em} \al_\sun \frac{\du}{Y} 
  = \int_{-\pi}^{+\pi} \!\! \frac{\atan2\,(\sin\la\,\cob, \cos\la)}{(1-e^2)^{-3/2}\,[1+e\cos(\la-\la_\peri)]^2} \frac{\dla}{2\pi} \\
& = \int_0^{\pi} \frac{\atan2\,(\sin\la\,\cob, \cos\la)}{(1-e^2)^{-3/2}}D(\la; \la_\peri) \frac{\dla}{2\pi},
\end{split}
\eeq
since $\atan2$ is odd in $\la$, where
\beq
\begin{split}
D(\la; \la_\peri) 
& \equiv \frac{1}{[1+e\cos(\la - \la_\peri)]^2}  - \frac{1}{[1+e\cos(\la + \la_\peri)]^2} \\
& = \frac{[1+e\cos(\la + \la_\peri)]^2 - [1+e\cos(\la - \la_\peri)]^2}{[(1+e\cos(\la - \la_\peri))(1+e\cos(\la + \la_\peri))]^2} \\
& = \frac{2e(\cos(\la + \la_\peri) - \cos(\la - \la_\peri)) + e^2(\cos^2(\la + \la_\peri) - \cos^2(\la - \la_\peri))}{[1 + e(\cos(\la - \la_\peri) + \cos(\la + \la_\peri)) + e^2 \cos(\la - \la_\peri) \cos(\la + \la_\peri)]^2} \\
& = \frac{-4e\sin\la\sin\la_\peri - 4\,e^2\cos\la\cos\la_\peri\sin\la\sin\la_\peri }{[1 + 2e\cos\la\cos\la_\peri + e^2 (\cos^2\!\la\,\cos^2\!\la_\peri - \sin^2\!\la\,\sin^2\!\la_\peri)]^2} \\
%& = \frac{ -4 e \sin\la\sin\la_\peri (1 + e\cos\la\cos\la_\peri) }{[1 + 2e\cos\la\cos\la_\peri + e^2 (\cos^2\!\la\,\cos^2\!\la_\peri - \sin^2\!\la\,\sin^2\!\la_\peri)]^2}. \\
& = \frac{ -4 e \sin\la\sin\la_\peri (1 + e\cos\la\cos\la_\peri) }{[(1 + e\cos\la\cos\la_\peri)^2 - e^2 \sin^2\!\la\,\sin^2\!\la_\peri)]^2}.
\end{split}
\eeq
%\[ \cos(\la \pm \la_\peri) = \cos\la\cos\la_\peri \mp \sin\la\sin\la_\peri \]
%\[ \cos^2(\la \pm \la_\peri) = \cos^2\!\la\,\cos^2\!\la_\peri \mp 2\cos\la\cos\la_\peri\sin\la\sin\la_\peri + \sin^2\!\la\,\sin^2\!\la_\peri \]
%\[ \cos(\la + \la_\peri)\cos(\la - \la_\peri) = \cos^2\!\la\,\cos^2\!\la_\peri - \sin^2\!\la\,\sin^2\!\la_\peri \]
Continuing with the reduction,
\beq
\begin{split}
\langle& \al_\sun\rangle_{u_{\fpoa'}+Y/2} = \int_0^{\pi/2} \frac{\atan2\,(\sin\la\,\cob, \cos\la)}{(1-e^2)^{-3/2}}D(\la; \la_\peri) \\
& \hspace{3cm} + \frac{\atan2\,(\cos\la\,\cob, -\sin\la)}{(1-e^2)^{-3/2}}D(\la\!+\!\pi/2; \la_\peri) \:\frac{\dla}{2\pi} \\
& = \int_0^{\pi/2} \hspace{-1em} \arctan(\tan\la\,\cob) \frac{D(\la; \la_\peri)}{(1-e^2)^{-3/2}} + [\pi-\arctan(\cot\la\,\cob)] \frac{D(\la\!+\!\pi/2; \la_\peri)}{(1-e^2)^{-3/2}} \:\frac{\dla}{2\pi},
\end{split}
\eeq
where
\beq
D(\la+\pi/2; \la_\peri) = \frac{ -4 e \cos\la\sin\la_\peri (1 - e\sin\la\cos\la_\peri) }{[(1 - e\sin\la\cos\la_\peri)^2  - e^2 \cos^2\!\la\,\sin^2\!\la_\peri)]^2}.
\eeq

We will attempt a solution by expanding in powers of $e$, since $e \approx 0.0167 \ll 1$. 
Clearly for $e=0$ both $D$ terms are zero and we get our earlier special case result. 

\subsubsection{First order in $e$}

To {\em first\/} order in $e$:
\beq
\frac{D(\la; \la_\peri)}{(1-e^2)^{-3/2}} \approx -4 e \sin\la\sin\la_\peri, \quad
\frac{D(\la+\pi/2; \la_\peri)}{(1-e^2)^{-3/2}} \approx -4 e \cos\la\sin\la_\peri,
\eeq
and so
\beq
\begin{split}
\langle& \al_\sun\rangle_{u_{\fpoa'}+Y/2} \\
& \approx -4 \, e \sin\la_\peri \int_0^{\pi/2} \hspace{-1em} \arctan(\tan\la\,\cob) \sin\la + [\pi-\arctan(\cot\la\,\cob)] \cos\la \:\frac{\dla}{2\pi} \\
& = \frac{-e \sin\la_\peri}{\pi/2} \Big[ 
\cot\eps \artanh(\sin\la\,\sob) - \cos\la \arctan(\tan\la\,\cob) \\
& \hspace{1.75cm} + \cot\eps \artanh(\cos\la\,\sob) - \sin\la \arctan(\cot\la\,\cob) + \pi \sin\la \Big]_0^{\pi/2} \\
& = \frac{-e \sin\la_\peri}{\pi/2} \Big[ \cot\eps [\artanh(\sob) - \artanh(0)] - \{ 0 \arctan(\infty) - \arctan(0) \} \\
& \hspace{2cm} + \cot\eps [\artanh(0) - \artanh(\sob)] - \{ \arctan(0) - 0 \arctan(\infty) \} + \pi \Big] \\
& = -2e \sin\la_\peri.
\end{split}
\eeq
Now, by (\ref{eq:mEOT-gen}),
\beq
\begin{split}
\langle EOT & \rangle_{u_{\fpoa'}+Y/2} \\
& = \MA(u_{\fpoa'}+Y/2) + \la_\peri - \langle \al_\sun \rangle_{u_{\fpoa'}+Y/2} \\
& = \MA(u_{\fpoa'}) + \pi + \la_\peri + 2e \sin\la_\peri \\
& = E(u_{\fpoa'})  - e \sin E(u_{\fpoa'}) + \pi + \la_\peri + 2e \sin\la_\peri \\
& = 2\arctan\left[ \sqrt{\frac{1-e}{1+e}}\tan\left(\frac{\nu(u_{\fpoa'})}{2}\right) \right]  - \frac{e\sqrt{1-e^2}\sin\nu(u_{\fpoa'})}{1 + e\cos\nu(u_{\fpoa'})} + \pi + \la_\peri + 2e \sin\la_\peri \\
& = -2\arctan\left[ \sqrt{\frac{1-e}{1+e}}\tan\left(\frac{\la_\peri + \pi}{2}\right) \right] + \frac{e\sqrt{1-e^2}\sin(\la_\peri + \pi)}{1 + e\cos(\la_\peri + \pi)} + \pi + \la_\peri + 2e \sin\la_\peri \\
& = -2\arctan\left[ \sqrt{\frac{1-e}{1+e}}\tan\left(\frac{\la_\peri + \pi}{2}\right) \right] - \frac{e\sqrt{1-e^2}\sin\la_\peri}{1 - e\cos\la_\peri} + \pi + \la_\peri + 2e \sin\la_\peri,
\end{split}
\eeq
since $\MA = E - e\sin E$ and since the eccentric anomaly $E$ obeys the following relations from the two-body solution,
\beq
\sin E = \frac{\sqrt{1-e^2}\sin\nu}{1 + e\cos\nu}, \quad
\tan(E/2) = \sqrt{\frac{1-e}{1+e}}\tan(\nu/2),
\eeq
with $\nu = \la - \la_\peri$, and since $\nu(u_{\fpoa'}) = \la(u_{\fpoa'}) - \la_\peri = -(\la_\peri + \pi)$.
Hence, to our first order in $e$ approximation,
\beq
\begin{split}
\langle EOT & \rangle_{u_{\fpoa'}+Y/2} 
   \approx -2\arctan\left[ (1-e)\tan\left(\frac{\la_\peri + \pi}{2} \right) \right] - e\sin\la_\peri + \pi + \la_\peri + 2e \sin\la_\peri \\
& \approx -2 \arctan\left[ \tan\left(\frac{\la_\peri + \pi}{2} \right) \right] + \frac{2e\tan\left(\frac{\la_\peri + \pi}{2} \right)}{1+\tan^2\left(\frac{\la_\peri + \pi}{2} \right)} + \pi + \la_\peri + e \sin\la_\peri \\
& =  e\sin2\left(\frac{\la_\peri + \pi}{2} \right) + e \sin\la_\peri = e [ \sin(\la_\peri + \pi) + \sin\la_\peri ] = 0,
\end{split}
\eeq
as required.
%\[ E = 2\arctan\left( \sqrt{\frac{1-e}{1+e}}\tan(\nu/2) \right) \]
We could proceed to higher order in $e$ by this method, but first we will try a slightly different approach, 
using integration by parts, which will be turn out to be easier \ldots

\subsection{General case using integration by parts}

Alternatively, we can integrate by parts:
\beq
\begin{split}
\langle\al_\sun\rangle_{u_{\fpoa'}+Y/2} 
& = \int_{M(u_{\fpoa'})}^{M(u_{\fpoa'})+2\pi} \hspace{-1em} \al_\sun \frac{\dM}{2\pi} 
 = \frac{1}{2\pi}\! \left( \left[\al_\sun M \right]_{u_{\fpoa'}}^{u_{\fpoa'}+Y}  - \int_{-\pi}^{+\pi} M \,\text{d}\al_\sun \right) \\
& = M(u_{\fpoa'})+\pi -  \frac{1}{2\pi} \int_{-\pi}^{+\pi} M \,\frac{\text{d}\al_\sun}{\dla} \,\dla \\
& = M(u_{\fpoa'})+\pi -  \frac{1}{2\pi} \int_{-\pi}^{+\pi} \frac{(E - e \sin E) \,\cob\; \dla} {1 - \sin^2\!\la \,\sin^2\eps},
\end{split}
\eeq
where (again) $M \equiv \MA = E - e \sin E$, and since $\al_\sun(u_{\fpoa'}) = \pm \pi$ and
\beq
\frac{\partial}{\partial x} \atan2(y,x) = \frac{-y}{x^2 + y^2} \quad\text{and}\quad
\frac{\partial}{\partial y} \atan2(y,x) = \frac{x}{x^2 + y^2},
\eeq
and so
\beq
\frac{\dee \al_\sun}{\dla} 
= \frac{\dee}{\dla} \atan2\,(\sin\la\,\cob, \cos\la) = \ldots
% = \frac{-\sin\la\,\cob \cdot -\sin\la + \cos\la \cdot \cos\la\,\cob} {\cos^2\!\la + \sin^2\!\la \,\cos^2\eps} \\
% = \frac{\cob} {\cos^2\!\la + \sin^2\!\la \,\cos^2\eps} 
= .\frac{\cob} {1 - \sin^2\!\la \,\sin^2\eps}
\label{eq;dasdla}
\eeq
Hence, before making any approximation in the order of $e$, we have
\beq
\begin{split}
\langle EOT & \rangle_{u_{\fpoa'}+Y/2} = \MA(u_{\fpoa'}) + \pi +  \la_\peri - \langle \al_\sun \rangle_{u_{\fpoa'}+Y/2} \\
& = \la_\peri + \frac{1}{2\pi} \int_{-\pi}^{+\pi} \frac{(E - e \sin E) \,\cob\; \dla} {1 - \sin^2\!\la \,\sin^2\eps} \\
& = \la_\peri + \frac{1}{2\pi} \int_{-\pi}^{+\pi} 
\frac{\left[2\arctan\left( \sqrt{\frac{1-e}{1+e}}\tan\left(\frac{\nu}{2}\right) \right) - \frac{e\sqrt{1-e^2}\sin\nu}{1 + e\cos\nu} \right] \cob\; \dla}{1 - \sin^2\!\la \,\sin^2\eps},
\end{split}
\eeq
with $\nu = \la - \la_\peri$. Clearly this method avoids explicit calculation of $ \MA(u_{\fpoa'})$.

\subsubsection{First order in $e$}

To {\em first\/} order in $e$:
\beq
\begin{split}
\langle EOT & \rangle_{u_{\fpoa'}+Y/2} \\
& \approx \la_\peri + \frac{1}{2\pi} \int_{-\pi}^{+\pi} \frac{\left[2\arctan\left( (1-e)\tan\left(\frac{\la - \la_\peri}{2}\right) \right) - e\sin(\la - \la_\peri) \right] \cob\; \dla} {1 - \sin^2\!\la \,\sin^2\eps} \\
& \approx \la_\peri + \frac{1}{2\pi} \int_{-\pi}^{+\pi} \frac{\left[\la - \la_\peri - \frac{2e \tan\left(\frac{\la - \la_\peri}{2}\right)}{1 +  \tan^2\left(\frac{\la - \la_\peri}{2}\right)}
  - e\sin(\la - \la_\peri) \right] \cob\; \dla} {1 - \sin^2\!\la \,\sin^2\eps} \\
& = \la_\peri + \frac{1}{2\pi} \int_{-\pi}^{+\pi} \frac{\left[\la - \la_\peri - 2e\sin(\la - \la_\peri) \right] \cob\; \dla} {1 - \sin^2\!\la \,\sin^2\eps} \\ 
& = \la_\peri + \frac{1}{2\pi} \int_{-\pi}^{+\pi} \frac{\left[\la - \la_\peri - 2e(\sin\la\cos\la_\peri - \cos\la\sin\la_\peri) \right] \cob\; \dla} {1 - \sin^2\!\la \,\sin^2\eps} \\ 
& = \la_\peri - \frac{1}{2\pi} \int_{-\pi}^{+\pi} \frac{\left[\la_\peri - 2e\sin\la_\peri\,\cos\la \right] \cob\; \dla} {1 - \sin^2\!\la \,\sin^2\eps},
\end{split}
\eeq
after removing odd functions of $\la$ in the last line. By (\ref{eq;dasdla}) we can simplify this to
\beq
\begin{split}
\langle EOT & \rangle_{u_{\fpoa'}+Y/2}
= \la_\peri - \frac{\la_\peri}{2\pi} \int_{-\pi}^{+\pi} \dee \al_\sun + \frac{2e\sin\la_\peri}{2\pi} \int_{-\pi}^{+\pi} \frac{\cos\la \,\cob\; \dla} {1 - \sin^2\!\la \,\sin^2\eps} \\
& = \frac{2e\sin\la_\peri}{2\pi} \int_0^0 \frac{\cob\; \dee Y} {1 - Y^2 \,\sin^2\eps} = 0, \quad \text{where}\;Y \equiv \sin\la,
\end{split}
\eeq
so again we have our required result to first order in $e$.

\subsubsection{Higher orders in $e$}

At this point we have not been able to go to higher orders in $e$. 
Can cannot say whether these orders will yield zero contributions to mean $EOT$ or not.
Here's where I got so far, which perhaps suggest that the mean $EOT$ is only zero to first order \ldots

The following binomial series converge absolutely since $e \ll 1$:
\beq
(1-e^2)^{1/2} = 1 + A, \quad A \equiv \sum_{k=1}^{\infty} {1/2 \choose k} (-1)^k e^{2k} \sim O(e^2),
\eeq

\beq
(1-e^2)^{-1/2} = 1 + B, \quad B \equiv \sum_{k=1}^{\infty} {-1/2 \choose k} (-1)^k e^{2k} \sim O(e^2),
\eeq
and
\beq
(1 + e\cos\nu)^{-1} = 1 + C, \quad C \equiv \sum_{k=1}^{\infty} {-1\choose k} e^k\cos^k\nu \sim O(e).
\eeq
Then
\beq
\sqrt{\frac{1-e}{1+e}} = (1-e) (1 + B) = 1 - e + D, \quad D \equiv (1-e) B \sim O(e^2),
\eeq
and, 
\beq
\frac{\sqrt{1-e^2}}{1 + e\cos\nu} = (1 + A) (1 + C) = 1 + C + A + AC.
\eeq
So, to general order in $e$, 
\beq
\begin{split}
Q &\equiv 2\arctan\left( \sqrt{\frac{1-e}{1+e}}\tan\left(\frac{\nu}{2}\right) \right) - \frac{e\sqrt{1-e^2}\sin\nu}{1 + e\cos\nu} \\
& = 2\arctan\left( \tan\left(\frac{\nu}{2}\right) + (D - e) \tan\left(\frac{\nu}{2}\right) \right) - e(1 + C + A + AC)\sin\nu = \ldots \\
& = \nu + (D - 2e - eA) \sin\nu - e(1 + A)C\sin\nu \\
& \quad\quad + 2 \sum_{n=2}^{\infty} \arctan^{(n)} \left( \tan\left(\frac{\nu}{2}\right) \right) \frac{(D - e)^n}{n!} \tan^n \left(\frac{\nu}{2}\right).
\end{split}
\eeq
Hence,
\beq
\begin{split}
\langle & EOT \rangle_{u_{\fpoa'}+Y/2} = \la_\peri + \frac{1}{2\pi} \int_{-\pi}^{+\pi} \frac{Q \,\cob\; \dla}{1 - \sin^2\!\la \,\sin^2\eps} = \ldots \\
& = \frac{1}{2\pi} \sum_{n=2}^{\infty}  \frac{(D - e)^n}{n!} \int_{-\pi}^{+\pi} \frac{2\arctan^{(n)} \left( \tan\left(\frac{\la-\la_\peri}{2}\right) \right) \tan^n \left(\frac{\la-\la_\peri}{2}\right) \,\cob\; \dla}{1 - \sin^2\!\la \,\sin^2\eps} \\
& \quad\quad - \frac{e(1 + A)}{2\pi} \sum_{k=1}^{\infty} {-1\choose k} e^k \int_{-\pi}^{+\pi} \frac{ \cos^k(\la-\la_\peri) \sin(\la-\la_\peri) \,\cob\; \dla}{1 - \sin^2\!\la \,\sin^2\eps}.
\label{eq:mEOT-high}
\end{split}
\eeq
First look at the $\cos^k(\la-\la_\peri) \sin(\la-\la_\peri)$ terms:
\beq
\cos^k(\la-\la_\peri) \sin(\la-\la_\peri) = (C_1C_{1p} + S_1S_{1p})^k (S_1C_{1p} - C_1S_{1p}).
\eeq
where $C_n \equiv \cos(n\la)$, $S_n \equiv \sin(n\la)$ and the $p$ subscripts are for the $\la_\peri$ versions.
For $k=1$,
\beq
(C_1C_{1p} + S_1S_{1p})(S_1C_{1p} - C_1S_{1p}).
%S_1C_1(C_{1p}^2 - S_{1p}^2) - (C_1^2  - S_1^2) S_{1p} C_{1p} 
=  (S_2 C_{2p} - C_2 S_{2p}) / 2.
\eeq
The first term is odd in $\la$ and will integrate to zero, but the second term is even and will not be zero! So perhaps our result 
is only good to first order in $e$? The order $e^2$ component from (\ref{eq:mEOT-high}) for these terms is then:
\beq
\begin{split}
\frac{e^2}{2\pi} & \int_{-\pi}^{+\pi} \frac{ \cos(\la-\la_\peri) \sin(\la-\la_\peri) \,\cob\; \dla}{1 - \sin^2\!\la \,\sin^2\eps} \\
& -\frac{e^2\sin(2\la_\peri)}{4\pi} \int_{-\pi}^{+\pi} \frac{ \cos(2\la) \,\cob\; \dla}{1 - \sin^2\!\la \,\sin^2\eps}  \\
& \quad = \frac{e^2\sin(2\la_\peri)}{8\pi\sin^2\eps} \Big[ (\cos(2\eps) + 3) \arctan(\cob \tan\la) - 4 \la\,\cob \Big]_{-\pi}^{+\pi} \\
& \quad = -e^2 \cot\eps \csc\eps \sin(2\la_\peri).
\label{eq:mEOT-2nd-a}
\end{split}
\eeq
Likewise, the derivatives of $\arctan$ term of  (\ref{eq:mEOT-high}) for order $e^2$ is:
\beq
\begin{split}
 \frac{e^2}{2} & \frac{1}{2\pi} \int_{-\pi}^{+\pi} \frac{2\arctan^{(2)} \left( \tan\left(\frac{\la-\la_\peri}{2}\right) \right) \tan^2 \left(\frac{\la-\la_\peri}{2}\right) \cob\; \dla}{1 - \sin^2\!\la \,\sin^2\eps} \\
% & = -\frac{e^2}{2} \frac{1}{2\pi} \int_{-\pi}^{+\pi} \frac{\left[ \frac{2\tan\left(\frac{\la-\la_\peri}{2}\right)}{1+\tan^2\left(\frac{\la-\la_\peri}{2}\right)}  \right]^2 \tan \left(\frac{\la-\la_\peri}{2}\right) \,\cob\; \dla}{1 - \sin^2\!\la \,\sin^2\eps} \\
& = -\frac{e^2}{2} \frac{1}{2\pi} \int_{-\pi}^{+\pi} \frac{\sin^2(\la-\la_\peri) \tan \left(\frac{\la-\la_\peri}{2}\right) \cob\; \dla}{1 - \sin^2\!\la \,\sin^2\eps} \\
& = -\frac{e^2}{2} \frac{1}{2\pi} \int_{-\pi}^{+\pi} \frac{\sin(\la-\la_\peri) [1 - \cos(\la-\la_\peri)] \,\cob\; \dla}{1 - \sin^2\!\la \,\sin^2\eps} \\
& = \frac{1}{2} \frac{e^2}{2\pi} \int_{-\pi}^{+\pi} \frac{\sin(\la-\la_\peri) \cos(\la-\la_\peri) \,\cob\; \dla}{1 - \sin^2\!\la \,\sin^2\eps},
\label{eq:mEOT-2nd-b}
\end{split}
\eeq
which is exactly half of (\ref{eq:mEOT-2nd-a}). So, we conclude that to second order in $e$, 
\beq
\langle EOT \rangle_{u_{\fpoa'}+Y/2} = -\tfrac{3}{2} e^2 \cot\eps \csc\eps \sin(2\la_\peri)
\eeq
We will not pursue any higher order terms. For typical J2000 values $e \approx 0.0167$, $\eps \approx 23.44^\circ$, 
and $\la_\peri \approx 102.95^\circ - 180^\circ = -77.05^\circ$,
the above formula gives
\beq
\langle EOT \rangle_{u_{\fpoa'}+Y/2} 
%\approx -1.5\, (0.0167)^2 \times5.798 \times -0.4368 
\approx 0.00106 \,\text{rad} \approx 15\,\text{sec}.
\eeq

Thus, we conclude that our simple choice of $u_R$ in (\ref{eq:uR-choice}) leads to zero mean $EOT$ only to first order in $e$.

\end{document}



