%
% This is the !DESCRIPTION entry for MAPL_CFIO.
%

      {\tt MAPL\_CFIO} provides {\em Climate and Forecast} (CF)
      compliant I/O methods for high level ESMF data types by using the
      CFIO Library. It currently includes read-write support for ESMF
      Bundles and States, and read-only support for ESMF Fields and
      Fortran arrays. The API consists of 4 basic methods:
\bi
      \item MAPL\_CFIORead
      \item MAPL\_CFIOCreate
      \item MAPL\_CFIOWrite
      \item MAPL\_CFIODestroy
\ei 

      {\bf Reading a file.}  When reading data from a CFIO compliant file, 
      in the very least the user needs to specify the file name, the time, 
      and an ESMF object to receive the data. There is no need to explicitly
      involve the {\tt MAPL\_CFIO} object in this case. For example, assuming that 
      one has already defined an ESMF grid and clock, here is how to
      read a single time instance of all variables from a file into an
      ESMF Bundle: 
\bv
      bundle = ESMF_BundleCreate ( name='Mary', grid=grid )
      call MAPL_CFIORead ( 'forecast_data.nc', clock, bundle )
\end{verbatim}
      This method will read all variables on file, doing any necessary 
      (horizontal) regriding to the ESMF {\tt grid} used to create the
      bundle, and allocating memory for each variable, as necessary. 
      Currently, the file is open, read from and subsequently closed.
      {\tt MAPL\_CFIO} also provides methods to read single variables into a
      simple fortran arrays:
\bv
      real :: ps(im,jm)
      call MAPL_CFIORead ( 'surfp', 'forecast_data.nc', clock, grid, ps )
\end{verbatim}
      This method will read the variable named 'surfp' into the 2D Fortran
      array {\tt ps}, performing any necessary (horizontal) interpolation
      to the destination {\tt grid}. Consult the API reference below
      for several optional parameters to the MAPL\_CFIORead() method,
      including the ability to select the variables to read and
      transparently perform time interpolation.

      {\bf Writing to a file.} For writing, a new file is created, written to,
      and explicitly closed. Assuming one has already defined an ESMF {\tt bundle} 
      and {\tt clock} here is how to save a bundle to a new file:
\bv
      type(MAPL_CFIO) :: mcfio
      call MAPL_CFIOcreate  ( mcfio, 'climate_data', clock, bundle )
      call MAPL_CFIOwrite   ( mcfio, clock )
      call MAPL_CFIOdestroy ( mcfio )
\end{verbatim}
      Consult the API reference below for many optional parameters controling
      the behavior of these methods.  As of this writing, a {\tt MAPL\_CFIOopen()} 
      function to write to an already existing file has not been implemented. 

      {\bf File formats.} {\tt MAPL\_CFIO} is designed to work with a variety of
      file formats, provided these files can be annotaded with the necessary
      CF metadata. The particularities of the specific file format is handled
      by the backend CFIO library. As of this writing the backend CFIO library 
      supports self-describing formats such as NetCDF and HDF, and support for
      GrADS compatible binary files is in alpha testing. There are
      also plans to add support for GRIB versions 1, 2 or both.

      {\bf Self-desbring (SDF) formats.} The support for SDF formats is implemented
      in the backend CFIO library using the NetCDF Version 2 API. This API is
      currently supported by NetCDF versions 2 through 4 and HDF version 4.
      By selecting one of these libraries at build time it is possible to read/write
      several versions of NetCDF/HDF as summarized in the following table:
\bv
        Library        Reads           Writes 
      -----------  -------------   ---------------- 
      HDF-4        NetCDF, HDF-4   HDF-4          
      NetCDF-2     NetCDF          NetCDF
      NetCDF-3     NetCDF          NetCDF
      NetCDF-4     NetCDF, HDF-5   NetCDF, HDF-5 
\end{verbatim}
      NetCDF versions 2 and 3 can only read/write its own native {\tt NetCDF}
      format. HDF version 4 offers some form of interoperability with NetCDF,
      but it can only write HDF-4 files. The new NetCDF version 4 is written
      on top of the HDF-5 library. This version of NetCDF still retains the
      ability of reading and writing its legacy {\tt NetCDF} format,
      but advanced features such as compression is only available when
      writing in HDF-5 format. Beware that NetCDF-4 can only read
      the particular kind of HDF-5 files it writes; it cannot read generic
      HDF-5 files such has HDF-5 EOS. Because the standard HDF-5 
      library (without NetCDF-4) no longer supports the NetCDF 2 API (or the 
      native HDF-4 API for that matter) it cannot be used with the SDF
      backend of the CFIO library.  It is important to notice that because
      of conflicts in the API one cannot load more than one NetCDF or HDF
      library in one application. 

      {\bf API Design Issues.} The {\tt MAPL\_CFIO} package is still under active
      development. The current state of the API was dictated by the
      features needed to build the GEOS-5 model at NASA/GSFC, and some
      asymmetry still remains in the API. In particular, the {\em read methods}
      utilize file names to specify the file object, while the {\em
      write methods} uses the {\tt MAPL\_CFIO} object much like a file
      handle. Both methods of access are valid and useful under different
      circunstances, and ought to be supported for both read and
      write operations. When using {\em file name} access mode the following should 
      be possible:
\bv
      call MAPL_CFIORead  ( 'forecast_data.nc', clock, bundle )
      call MAPL_CFIOWrite ( 'new_file.nc', clock, bundle )
      call MAPL_CFIOWrite ( 'existing_file.nc', clock, bundle, append=.true. )
\end{verbatim}
      In this case, the file is opened, read from/written to and closed 
      on exit. For users desiring to keep files open in between operations
      a {\em file handle} mode should be provided for both read and write. Here is a 
      typical use case for reading: 
\bv      
      mcfio = MAPL_CFIOopen ( 'forecast_data.nc' )
      call MAPL_CFIORead    ( mcfio, clock_now,   bundle )
      ...
      call MAPL_CFIORead    ( mcfio, clock_later, bundle )
      call MAPL_CFIOdestroy ( mcfio )
\end{verbatim}
      Future versions of {\tt MAPL\_CFIO} may support these features.


