
\def\m#1{{\tt \_\_#1\_\_}}
\def\f#1#2{{\tt \_\_#1\_\_(#2)}}

\subsection*{Introduction}

\paragraph{Definitions.} In Object Oriented languages such as C++, Java or
Python, the term {\em exception} is used to describe an event, which
occurs during the execution of a program, that disrupts the normal
flow of the program. The simple macros presented here attempt to
simulate some of the exception handling functionality within the
constraints of a Fortran programming environment.

When an error occurs in a MAPL or ESMF subroutine or function, the
subprogram usually returns with a non-zero error condition. It is
common practice in MAPL applications to {\em bubble up} this error
condition, , returning this error to all subprograms in the call
stack. Creating an exception by means of a nonzero error code and
handing it to the MAPL runtime system is called {\em raising an exception}
(as in Python; the term {\em throwing an exception} is used in Java.)

As the nonzero error condition travels up the call stack, one of these
subprograms may examine this error condition and decide to take
appropriate action.  In this case, we say that the subprgram {\em
catches} the exception. However, if each subprogram simply returns,
passing up the same nonzero error condition, when control reaches the
top of the call stack the program terminates.

\paragraph{Advantages of Exceptions.} Using the concept of exceptions
to manage errors has some advantages over traditional error-management
techniques. You can learn more in the {\em Advantages of Exceptions}
section in the Java tutorial:
\begin{verbatim}
http://java.sun.com/docs/books/tutorial/essential/exceptions/advantages.html
\end{verbatim}
These advantages also apply to Fortran codes, specifically,
\begin{description}

\item[Separating Error-Handling Code from Regular Code.] Exceptions
  provide the means to separate the details of what to do when
  something out of the ordinary happens from the main logic of a
  program. In traditional programming, error detection, reporting, and
  handling often lead to confusing spaghetti code.
  
\item[Propagating Errors Up the Call Stack].  A second advantage of
  exceptions is the ability to propagate error reporting up the call
  stack of subprograms, producing a traceback in the process,
  something that Fortran compilers are not good at.

\item[Grouping and Differentiating Error Types.] Although exceptions
  can be raised in very diverse circunstances, they are assigned
  generic codes such as {\tt ESMF\_RC\_ARG\_SIZE} which can help you
  decide how to handle such an exception in the calling
  subprograms. Notice that an {\em exception code} is meant to
  describe generic error categories, not necessarily something
  particular to a given subprogram.
\end{description}

\subsection*{Usage}

The {\tt MAPL\_Exceptions} function is implemented by means of C
pre-processor macros defined in {\tt MAPL\_Exceptions.h}. This must be
the first file included at the top of your source code file:
\begin{verbatim}
 #include "MAPL_Exceptions.h"
\end{verbatim}
Another possibility is to include
\begin{verbatim}
 #include "MAPL.h"
\end{verbatim}
which implicitly includes {\tt MAPL\_Exceptions.h}. 

\paragraph{Important Remark.} Never include {\tt MAPL\_Exceptions.h}
after {\tt MAPL\_ErrLog.h} or a compilation error will result. 

\paragraph{Reserved variables: {\tt STATUS} and {\tt Iam}.} Most
macros in {\tt MAPL\_Exceptions.h} assume that the following
(reserved) variables have been locally declared:
\begin{verbatim}
    integer                     :: STATUS
    character(len=ESMF_MAXSTR)  :: Iam
\end{verbatim}
For convenience, the \f{Iam}{name} macro-function is provided as a
short-hand and should be invoked at the declaration section of your
subprogram. The string {\tt name} is used to initialize the variable
{\tt Iam}. For example,
\begin{verbatim}
    type(ESMF_Grid) :: Grid
    __Iam__('Initialize')
\end{verbatim}
is equivalent to
\begin{verbatim}
    type(ESMF_Grid) :: Grid
    integer                     :: STATUS
    character(len=ESMF_MAXSTR)  :: Iam='Initialize'
\end{verbatim}
Of course, you can redefine {\tt Iam} later if you wish.

\paragraph{Automatic handling of exceptions.}

Both MAPL and the ESMF use the optional variable {\tt RC} to return
the {\em error code} associated with a subprogram. It is common
practice to specify {\tt RC} as the the last argument of the subprogram. MAPL
applications using the lower level {\tt MAPL\_ErrLog} rely on the
macro-function {\tt VERIFY\_()} for exception handling. {\tt
  VERIFY\_(STATUS)} checks the value {\tt STATUS}, and in case of an
abnormal STATUS (really, {\tt STATUS/=ESMF\_SUCCESS)}, prints out the line
number and the contents of the string {\tt Iam}, returning to the caller, e.g.,
\begin{verbatim}
   call ESMF_ConfigGetAttribute(CF, provider, Label=Label, RC=STATUS)
   _VERIFY(STATUS)
\end{verbatim}
This error code {\em bubbles up} to the the main program, producing a
traceback in the process.

As a form of {\em syntatic sugar}, {\tt MAPL\_Exceptions} provide the
macro \m{RC} for accomplishing exactly the same without the need for
the extra {\tt VERIFY\_(STATUS)} line:
\begin{verbatim}
   call ESMF_ConfigGetAttribute(CF, provider, Label=Label, __RC__)
\end{verbatim}
Make sure that \m{RC} always comes in place of the {\em last} argument.
Another convenience macro is \m{STAT} which can be used with {\tt
  allocate statements}, viz.
\begin{verbatim}
   call allocate(uwnd(im,jm,lm), __STAT__)
\end{verbatim}
as a short-hand for
\begin{verbatim}
   call allocate(uwnd(im,jm,lm), stat=STATUS)
   _VERIFY(STATUS)
\end{verbatim}



\paragraph{Simple exception handling with \m{try}. }

The \m{try} macro allows you to define a block of code that executes
until the end of the block or up to the line where an abnormal error
condition occurs. Take a look at this code fragment,
\begin{verbatim}
 __try__
    AERO_PROVIDER = GetProvider_(CF,'AERO_PROVIDER:',__rc__)
    call ESMF_StateGetBundle(EXPORT, 'AERO', xBUNDLE, __rc__ )
    call ESMF_BundleGet(xBUNDLE, fieldCount=NA, __rc__ )
    call ESMF_StateGetBundle (GEX(AERO_PROVIDER), 'AERO', cBUNDLE, __rc__ )
    call ESMF_BundleGet(cBUNDLE, fieldCount=pNA, __rc__ )
__except__
    AERO_PROVIDER = PCHEM
__endtry__
\end{verbatim}
(Notice the lower case macro \m{rc}, which is not to be confused with
\m{RC} used above for automatic exception handling.) For the sake of
argument, let's say that an error occurs at the third line
\begin{verbatim}
    call ESMF_BundleGet(xBUNDLE, fieldCount=NA, __rc__ )
\end{verbatim}
then the subsequent lines will not be executed, with execution
resuming after \m{except} until an \m{endtry} is found. The integer
variable {\tt STATUS} holds the value of the last abnormal error condition.

\paragraph{VERY IMPORTANT.} In the current implementation, you cannot
have Fortran do-loops inside a \m{try}/\m{endtry} block. Use the
alternative construct \m{Try}/\m{endTry} in such cases.

\paragraph{Catching multiple exceptions.}

The example above {\em catches} all the exceptions under the
\m{except} block. Use the \m{catch}/\m{endcatch} macros when you need
to catch multiple exceptions as in this example:
\begin{verbatim}
 __try__
    AERO_PROVIDER = GetProvider_(CF,'AERO_PROVIDER:',__rc__)
    call ESMF_StateGetBundle(EXPORT, 'AERO', xBUNDLE, __rc__ )
    call ESMF_BundleGet(xBUNDLE, fieldCount=NA, __rc__ )
    call ESMF_StateGetBundle (GEX(AERO_PROVIDER), 'AERO', cBUNDLE, __rc__ )
    call ESMF_BundleGet(cBUNDLE, fieldCount=pNA, __rc__ )
 __endtry__
 
    __catch__
       case (MAPL_RC_RESOURCE_NOT_FOUND) 
                                          AERO_PROVIDER = PCHEM
       case (ESMF_RC_CANNOT_GET) 
                                          AERO_PROVIDER = GOCART
       case (ESMF_RC_MEM)
                                          AERO_PROVIDER = GMI
       case DEFAULT
                                          AERO_PROVIDER = NONE
   __endcatch__
\end{verbatim}
The \m{catch}/\m{endcatch} block is nothing more than a straighforward
wrapper around the Fortran {\tt select case/end select}
construct. These macros simply offer syntatic sugar to make it clear
that one is catching exceptions associated with the previous \m{try} block.

\paragraph{Named exception blocks with \m{Try}. } The current
implementation use Fortran do-loops with the \tt {exit} statement 
for implementing the \m{try} blocks. This prevents user do-loops
inside these blocks as the {\tt exit} would jump out of the user
do-loop, not the do-loop associated with exception block. In such
cases, use the alternative named \m{Try} blocks:
\begin{verbatim}
     __Try__(p)
        do i = 1, nLabel
          call ESMF_ConfigGetAttribute(CF, provider, Label=Lab(i), __Rc__(p) )
          select case ( trim(provider) )
           case ('PCHEM')
                                    GetProvider_ = PCHEM
           case ('GOCART')
                                    GetProvider_ = GOCART
           case ('STRATCHEM')
                                    GetProvider_ = STRATCHEM
           case ('GMI')
                                    GetProvider_ = GMI
           case DEFAULT
                __raise__(MAPL_RC_ERROR,'unknown provider '//trim(provider))
          end select
        end do
     __endTry__(p)
\end{verbatim}
Notice the use of \f{Rc}{p} to specify that when an exception
occurs the control should jump out of the \m{Try} block named {\tt
  p}. Currently there is no equivalent of \m{except} for named
exception blocks. However, \m{catch} works as usual.

\paragraph{Raising exceptions.} Use the \f{raise}{name,description}
macro-function to raise an exception, printing along an explanatory
error message. For example, if your code says
\begin{verbatim}
   __raise__(ESMF_RC_ARG_RANK,'2D arrays are required here')
\end{verbatim}
the following message will be printed:
\begin{verbatim}
ESMF_RC_ARG_RANK: 2D arrays are required here
\end{verbatim}
followed by the regular traceback information.

\paragraph{Error codes.} {\tt MAPL\_Exceptions} automatically includes
the file {\tt ESMF\_ErrReturnCodes.inc} which defines all possible
error codes returned by the ESMF. Consult the Appendix of the {\em ESMF
Reference Manual for Fortran} for a list of these error codes. In
addition, a handful of MAPL specific error codes are defined,
\begin{verbatim}
   MAPL_RC_ERROR                     2000
   MAPL_RC_RESOURCE_NOT_FOUND        2001
   MAPL_RC_FIELD_NOT_FOUND           2002
   MAPL_RC_BUNDLE_NOT_FOUND          2003
   MAPL_RC_FORTRAN_ARRAY_NOT_FOUND   2004
\end{verbatim}
Please examine the contents of {\tt MAPL\_Exceptions.h} for the most up
to date list of these error codes.





